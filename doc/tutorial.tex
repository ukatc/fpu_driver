\documentclass{scrartcl}[12pt,a4paper]
%\documentclass{article}[12pt,a4paper]
%\documentclass{artikel3}[12pt,a4paper]

\usepackage{minted}
\usepackage{hyperref}


\begin{document}

\title{Tutorial for the MOONS FPU grid driver Python library
  module} \subtitle{For driver version 0.5.0}

\author{Johannes Nix}

\maketitle

\tableofcontents


\section{Introduction}
\subsection{Topic and goals}
This tutorial describes how to use the Python bindings for the MOONS
fibre positioner unit (FPU) driver. Its goal is to enable the reader
to control the FPU in the verification system using python scripts, so
that he or she can perform basic verification tasks.


\subsection{Required knowledge}

It is assumed that the reader has basic knowledge of either Python or
a similar object-oriented language, like Java or c\#.  In the case
that using Python and scripts is completely new to the reader, it is
probably helpful to consult the tutorial for Python 2.7 at
\url{https://docs.python.org/2.7/tutorial/}.  Both the current version
of the driver as well as this document use Python 2.7, which has some
differences to Python 3. However this introduction does so in a way
that differences to Python 3 are minimised.

Also, in order to retrieve the current version of the library,
the user needs to have a minimum knowledge of the git version
control system. This is explained, for example, in \url{https://git-scm.com/book/en/v2/Git-Basics-Getting-a-Git-Repository}.

\subsection{Setting up the library module}

\subsubsection{Getting the source code}

To set up the library module, it is first
necessary to retrieve a current copy of its source code.
On the moons-pc01 workstation, this can be done using the
command \mint{bash}|git clone /home/jnix/fpu_driver|.
This creates a new directory with the name \texttt{fpu\_driver}
and populates it with the source code.

Alternatively, a fresh copy can be retrieved from the UKATC dalriada
server. Doing this this requires a personal user account on that
server. Assuming the account name is \texttt{carl}, this is done by
the command

\mint{bash}|git clone carl@dalriada/sw/sw4/jnix/MOONS/fpu_driver|

After entering the password for user \texttt{carl}, the
source code is copied into the new folder \texttt{fpu\_driver}.

\subsubsection{Refreshing the source code}
To refresh the source code, change the
working directory to \texttt{fpu\_driver},
and issue the command

\mint{bash}|git pull|

\subsubsection{Building the library module}

To build the Python library module,
change the working directory to \texttt{fpu\_driver},
and issue the command \mint{bash}|make pyext|

In order for this to work, the required dependencies need to be
installed. For the MOONS workstation, this is always the case. To
install these dependencies on other computers, follow the detailed
instructions in the file ``\texttt{doc/INSTALL}''.

\subsubsection{Environment configuration}

For the Python module to work, the directory \texttt{/user/local/lib}
needs to be added to the environment variable
\texttt{LD\_LIBRARY\_PATH}.  Assuming a standard setup, this can be
done by adding the following line to the file \texttt{.bashrc} in the
home directory:

\texttt{export LD\_LIBRARY\_PATH=\$LD\_LIBRARY\_PATH:/usr/local/lib}.

All the examples presented here assume that the current working
directory is \texttt{fpu\_driver/python}. If this is not the case,
this directory needs to be added to the \texttt{PYTHONPATH}
environment variable.


\subsection{Communication protocol versions}

The FPU driver sends commands to the fibre positioner units, which use
a specific communication protocol.  The current version of the driver
(version 0.5.0) uses communication protocol version 1, which has some
restrictions in functionality. For example, some edge cases cannot be
checked, some operations might take longer, and some errors are not
recognised and cannot be easily worked around. The main significant
difference which is visible to the users of the FPU driver module is
that cases in which the alpha arm of the FPU hits the limit switch
need to be manually recovered from. This is explained in
section\ref{sec:recovery} on page \pageref{sec:recovery}.


\section{A minimal example}

The following listing shows a minimal example of the
driver usage, and explains the underlying concepts
step by step.

\begin{minted}[linenos]{python}

from __future__ import print_function
import FpuGridDriver
from FpuGridDriver import TEST_GATEWAY_ADRESS_LIST
from fpu_commands import *

NUM_FPUS = 1
gd = FpuGridDriver.GridDriver(NUM_FPUS)

print("initializing driver: ", gd.initializeDriver())
print("connecting grid:", gd.connect(address_list=TEST_GATEWAY_ADRESS_LIST))


print("getting grid state:")
grid_state = gd.getGridState()


print("issuing findDatum:")
gd.findDatum(grid_state)
print("findDatum finished")

# We use grid_state to display the starting position
print("the starting position (in degrees) is:", list_angles(grid_state))


# Generate a waveform which moves the alpha arm by +45
# degree, and the beta arm by +15 degree. 

alpha_move = 45
beta_move = 15

# Generate the required waveform for one FPU
waveform = gen_wf(alpha_move, beta_move)

# upload the waveform to the FPU
gd.configMotion(waveform, grid_state)

# start the movement
print("starting movement by (45,15) degree")
gd.executeMotion(grid_state)

print("the reached position (in degrees) is:", list_angles(grid_state))

print("ready")

\end{minted}


\subsection{Import statements}

The import statements in lines 1 -- 4 load the following modules and
configurations:

\begin{itemize}
\item In line 1, the print function and the division operator is
  configured to work as in Python 3.
  
\item ``\texttt{import FpuGridDriver}'' in line 2 loads the driver
  object.

\item The line ``\texttt{from FpuGridDriver import
  TEST\_GATEWAY\_ADRESS\_LIST}'' loads a default address which refers to
  the EtherCAN gateway which is used. This address will be different
  when one wants to select a different gateway.

\item The line ``\texttt{from fpu\_commands import *}'' imports a few
  utility commands which help to display positions and to generate
  movement waveforms.

 
\end{itemize}

\subsection{Grid driver object}

In line 6, the used number of FPUs is set to one, and in line 7, the
FPU grid driver is initialised with that value. The returned object,
which is referenced by the Python variable \texttt{gd}, is always
required to access the FPUs.

\subsection{Driver methods}

The driver objects allows to issue method calls. The method names
always start with a dot after the name of the driver object.  So,
\texttt{gd.abc()} would call method ``abc'' of the driver object. Like
normal function calls, they usually have parameters. More concretely,
two methods are used to set up the driver:

\begin{description}
\item[\texttt{initializeDriver()}] initialises the driver object so
  that it can be used. This method needs to be called always at
  beginning.

\item[\texttt{connect()}] connects the driver to one or more EtherCAN
  gateways and starts listening to messages from the FPUs. The
  parameter defines to \emph{which} gateways the driver listens to.

\end{description}

As can be seen in the example, the methods have return values.  In
case of serious errors, the methods raise exceptions, as is the style
in Python.

the Python program does not need to explicitly disconnect
and de-initialise the driver, this happens automatically
when the program terminates\footnote{However if required, there
are methods which do that, too!}.



\subsection{Grid state variable}

Almost all commands to the driver return a snapshot of the current
state of the used FPU (or, if there are several, of all of them). That
snapshot is passed in the variable \texttt{grid\_state}. This variable
contains all the information a user might need about the state of the
FPUs - their current positions, which commands they will accept,
whether there are any collisions, whether the step counters have been
zeroed, whether a limit switch was hit, and so on. With each
invocation of a method, the old grid state is passed as a parameter,
and its new value is returned in the same variable.

To retrieve a current grid state information, the method
``getGridState()'' is used, as is done in line 14. In our listing, the
grid state variable is assigned to the variable \texttt{grid\_state}.
The \texttt{getGridState()} function always returns immediately, as it
just takes a snapshot. Because \texttt{grid\_state} is used so
frequently, a short-hand which many other programs use is the name
\texttt{gs}.

\subsection{Moving the FPU around}

The program then proceeds to use three methods to move
the FPUs:

\begin{description}
\item[\texttt{findDatum()}] moves the FPU to the datum position. After this,
  the internal step counters are set to zero, both for the alpha and
  for the beta arm.

\item[\texttt{configMotion()}] is a method which sends a table with movement
  operations, also called waveform table, to the FPU. In the listing,
  this is done in line 35. The movement table is stored in the
  variable \texttt{waveform}, which is defined in line 32.

\item[\texttt{executeMotion()}] is the command which starts the movement of the
  FPU. This command returns when the movement is complete and the FPUs
  have stopped to move. In case of an error, the command generates a
  Python exception.

\end{description}

\subsection{Utility functions}

The example also shows two utility functions for generating movement
data, and displaying information about the FPU state:

\begin{description}
\item[\texttt{list\_angles()}] is a function which takes a \texttt{grid\_state}
  variable, and displays the approximate positions of the FPUs as
  alpha and beta angles, in degree units.

\item[\texttt{gen\_wf()}] is a function which takes an (alpha,beta)
  value pair, and generates a valid waveform which can be send to the
  FPU.  The waveform which is passed is a regular Python data
  structure and can be displayed and manipulated like any other Python
  object.  The user only needs to be careful that it contains valid
  movement data when it is passed to the \texttt{configMotion()}
  method!

\end{description}

\subsection{Interactive inspection of the FPU state}

\label{sec:fpustate}
The grid state variable holds a large amount of detail information
about the current state of FPUs. This is by design, because
upper layers of the software need to be able to deal with
errors such as collisions, and correct them.

When trying to diagnose and understand errors, it is often helpful to
access this state information in an interactive way.
We can do this based on the property that Python can
run any commands which appear in a script equally well
when entered interactively. Also, the Python interpreter
can be started with the ``-i'' (interactive) option which
causes that after an error, or after all commands
in a script have been executed, the interpreter waits
for command line arguments.

For example, let's assume that the script above is shortened to the
first 36 lines and stored to a file named
\texttt{short\_script.py}. We then might run the following interactive
session (where the lines starting with ``\texttt{\$}'' and
``\texttt{>>>}'' are interactive input, and everything else is
output):



\begin{minted}{python}

  
$ python -i short_script.py
initializing driver:  DE_OK
connecting grid: DE_OK
getting grid state:
issuing findDatum:
findDatum finished
the starting position (in degrees) is: [(0.0, 0.0)]
>>> grid_state.FPU[0]
{ 'last_updated' : 61547082.491 'pending_command_set' : 0
  'pending_command_set' : 0 'state' : 'READY_FORWARD'
  'last_command' : 1 'last_status' : 0 'alpha_steps' : 0
  'beta_steps' : 0 'alpha_deviation' : 0 'beta_deviation' : 0
  'timeout_count' : 0 'step_timing_errcount' : 0
  'direction_alpha' : 0 'direction_beta' : 0
  'num_waveform_segments' : 0 'num_active_timeouts' : 0
  'sequence_number' : 0 'ping_ok' : 0 'movement_complete' : 0
  'was_zeroed' : 1 'is_locked' : 0 'alpha_datum_switch_active' : 0
  'beta_datum_switch_active' : 0 'at_alpha_limit' : 0
  'beta_collision' : 0 'waveform_valid' : 1 'waveform_ready' : 1
  'waveform_reversed' : 0 }
>>> list_angles(grid_state)
[(0.0, 0.0)]
>>> gd.executeMotion(grid_state)
fpu_driver.E_DriverErrCode.DE_OK
>>> grid_state.FPU[0]
{ 'last_updated' : 61547154.98 'pending_command_set' : 0
  'pending_command_set' : 0 'state' : 'RESTING'
  'last_command' : 7 'last_status' : 0 'alpha_steps' : 5625
  'beta_steps' : 1200 'alpha_deviation' : 0 'beta_deviation' : 0
  'timeout_count' : 0 'step_timing_errcount' : 0
  'direction_alpha' : 0 'direction_beta' : 0
  'num_waveform_segments' : 0 'num_active_timeouts' : 0
  'sequence_number' : 0 'ping_ok' : 1 'movement_complete' : 1
  'was_zeroed' : 1 'is_locked' : 0 'alpha_datum_switch_active' : 0
  'beta_datum_switch_active' : 0 'at_alpha_limit' : 0
  'beta_collision' : 0 'waveform_valid' : 1 'waveform_ready' : 0
  'waveform_reversed' : 0 }
>>> list_angles(grid_state)
[(45.0, 15.0)]
>>>
\end{minted}

In this listing, \verb+grid_state.FPU[0]+ displays the state of
FPU 0 (which is the only one we have because \texttt{NUM\_FPUS} was
set to 1).

Just to pin-point two of the more important bits of information, the
fields ``alpha\_steps'' and ``beta\_steps'' contain the current step
counts of an FPU, and the fields 'alpha\_datum\_switch\_active' and
'beta\_datum\_switch\_active' contain the current position of the
datum switches. The field \texttt{state} contains the current state of
the FPU. One can see that after issuing the \texttt{executeMotion}
command, the state of the FPU changes from \texttt{READY\_FORWARD} to
\texttt{RESTING}. Actually, the output of the \texttt{list\_angles()}
utility function is just a scaled version of the ``alpha\_steps'' and
``beta\_steps'' fields.




\section{Common tasks}

The following sections describe several common tasks
and which commands can be used to achieve them.


\subsection{Checking the connection}

At some points, it is of interest what the state of
the connection to the CAN gateway and the FPUs is.
This can be achieved by the following commands:
\begin{minted}{python}
grid_state = gd.getGridState()
grid_state.driver_state  
\end{minted}

The return value is the connection state of the driver, which is
\texttt{DS\_CONNECTED} of the connection is live.  In the case that the
connection fails, it is still possible to retrieve state information
on the FPUs, but it cannot be refreshed any more until the connection
is re-established using the \texttt{connect()} method of the driver
object.

Of course, it is possible that the connection between driver
and EtherCAN gateway works, but that the FPU does not
respond. This can be checked by the commands:

\begin{minted}{python}
  gd.pingFPUs(grid_state)
  grid_state.FPU[0].ping_ok
\end{minted}

In the case that the FPU responds, the returned value is
\texttt{True}.  If the FPU does not respond, the
value of the \verb+ping_ok+ is \texttt{False}.



\subsection{Getting the current position}

We already saw how to use the \texttt{pingFPUs()} method
of the driver object. This method automatically
retrieves the current step counters of the FPU.
there are two utility functions which display them:

\begin{minted}{python}
  list_positions(grid_state)
\end{minted}

shows the values of the alpha and beta step counters.
Often, it is interesting to know the value scaled
to an approximate angle. The scaled values
are displayed by the command

\begin{minted}{python}
  list_angles(grid_state)
\end{minted}


One can wonder how this angle is calibrated and
what is the relative position to the datum
position. The answer is, these values are
not calibrated - \texttt{list\_angles()}
shows only an approximately scaled value.
Also, the displayed values are only relative
to the datum position, which is defined to
have the step counter values (0,0). If
the datum position was not searched
at least once before, both values are meaningless.




\subsection{Moving the FPU to the datum position}

Because the FPU's step counters need to be zeroed
before any accurate positioning is possible, a
sensible operation early after powering on
the FPU is to move the FPUs to the datum position.
This is done by the command

\begin{minted}{python}
  gd.findDatum(grid_state)
\end{minted}

It is also highly recommend to move the FPUs to the datum position
before powering off. Doing this avoids to have the FPU in an ambiguous
state which might easily lead to hardware damage.

\subsection{Moving the FPU}

To move the FPU in a controlled way,
four steps are necessary:

\begin{enumerate}
\item Moving the FPU to the datum position, as explained above.

\item Generating a valid movement waveform table. For a single FPU,
  this is basically a list which defines for a number of short time
  segments how many steps the FPU stepper motors should move, and in
  which direction.

\item Sending the waveform to the FPU
\item Starting the movement
  
\end{enumerate}

There are two options for defining the waveform. The simpler one is to
automatically generate a waveform where we merely pass two values
which define how much the alpha and beta arms should move.

\subsubsection{Moving by an angle}

\begin{minted}[linenos]{python}
alpha_move = 45
beta_move = 15

# Generate the required waveform for one FPU
waveform = gen_wf(alpha_move, beta_move)

# upload the waveform to the FPU
gd.configMotion(waveform, grid_state)

# start the movement
print("starting movement by (45,15) degree")
gd.executeMotion(grid_state)
\end{minted}

The above listing shows how to generate a waveform which moves the FPU
by a certain (alpha,beta) angle difference.

Lines 1 and 2 define variables with the angles we want to move. In
line 5, a new waveform table called \texttt{waveform} is generated
using the \texttt{gen\_wf()} utility function.  By default, this
function generates the quickest valid movement by the angles
requested.

In line 8, the \texttt{configMotion()} method is used to send the
waveform to the FPU. As usual, the \texttt{grid\_state} variable is
passed, and the resulting state of the FPU is returned in this
variable.

Finally, in line 12, the method \texttt{executeMotion()} is called,
which starts the movement of the FPU (or, if we have more than one
FPU, the movement of all of them). In the normal case, this command
blocks and returns when the FPUs have finished moving.  In the case
that there is an error, for example a collision, a Python exception
will be raised.



\subsubsection{Manually defining a waveform table}

\paragraph{Waveform syntax}

In the example above, the waveform was generated by the
\texttt{gen\_wf()} utility function.  For normal movements, this is a
good option, and the actual value of the step counters can always be
retrieved using the \texttt{list\_positions()} utility
function. Sometimes, a much closer control of the FPU movements might
be required. This can be done with a code fragment like this:

\begin{minted}[linenos]{python}
wtable = { 0: [ ( 10, -20),
           ( 125, -125),
           ( 140, -140),
           ( 130, -125),
           ( 125, -50) ],
}
gd.configMotion(wtable, grid_state)
\end{minted}

Here, the Python variable wtable is assigned with
a waveform. The required format for a waveform
is \emph{a dictionary with a list of 2-tuples
  with the alpha and beta step counts}. Because
that sounds a bit complex, let's dissect this
structure into its components:

\begin{itemize}
  
\item At the top layer, we have a Python dictionary.  The key of each
  dictionary entry, and integer, is the numerical ID of the FPU we are
  addressing.  Because we only have one FPU, and the numbering starts
  with zero, the key is simply zero.

\item The corresponding value for the key zero is a Python list, as
  indicated by the brackets and a comma-separated sequence. The
  sequence is a list with exactly one entry for each time-slice of the
  movement operation\footnote{With the current default values,
  each time slice has a fixed duration of 250 milliseconds}.

\item Each list entry is a tuple with two elements, both of which are
  numbers. The first element is the number of alpha steps, and the
  second element is the number of beta steps.

  A positive number means that the FPU should move counter-clockwise
  (when viewed from above), and a negative number means it should move
  clock-wise (when viewed from above). If the value is zero, then the
  FPU will rest.

\end{itemize}

Now, we can decipher the above structure into ``in the first time
slice of the movement, the alpha arm should move ten steps
counter-clockwise, and the beta arm should move twenty steps
clockwise,'' and so on.

\paragraph{Waveform validity rules}

When you define waveforms, they have to conform to a
number of rules and conditions to make sure that the
FPU can execute them. Without attempting to give an
exhaustive list of these rules, the most important
ones are currently as follows:

\begin{enumerate}

\item All step numbers need to be integer values.
  
\item Except the first and the last non-zero entry,
  all step numbers in a waveform need to have
  an absolute value which is at least a minimum value of
  125.

\item No step number value should have an absolute value larger than
  the maximum limit of 500.

\item The number of waveform entries can only be 128\footnote{In
  version 2 of the firmware communication protocol, this limit is
  raised to a value of 256}.

\item Between consecutive step number values, the larger number can
  only be at most 40\% larger than the smaller number. The start and
  the end of the waveform, which are smaller than the minimum value of
  125, are exempt from this rule.

\item If a waveform has only two entries, it is not allowed that the
  first and the second entry are both below the minimum value.
  
\end{enumerate}



\subsection{Aborting a movement}

In the case that one wants to abort a \texttt{findDatum()} or a
\texttt{executeMotion()} operation, one can press
\verb+<Control-c>+. This terminates the movement in about 0.1
seconds. \footnote{This generates a \texttt{SIGINT} signal which is
  during the operation of both commands caught be the Python
  interpreter. The Python interpreter then sends an
  \texttt{abortMotion()} command to the driver object, which results
  in an \texttt{ABORT\_MOTION} CAN message.}

If it is necessary to abort a movement from Python -- for example,
from another Python thread -- this can be done using the
\texttt{abortMotion()} method of the grid driver object.  Like all
driver object methods, this method is thread-safe.

For other driver methods, a \verb+<Control-c>+ normally aborts the
current Python script when the method returns.\footnote{In some cases,
  it can happen that the Python program is stuck on a failed socket
  connection. Terminating such a connection can take considerable time
  because sockets are handled by the Linux kernel and by default, the
  kernel tries extremely hard to keep and re-animate an unreliable
  connection, even if the physical link is temporarily broken. To
  terminate such a hung program, it can be suspended using
  \texttt{<Control-z>} and then terminated using the UNIX
  \texttt{kill} shell command, or using
  \texttt{<Control-\textbackslash>}, which generates a
  \texttt{SIGQUIT} signal.}


\subsection{Reversing a movement}
Often, an FPU has been moved away from the datum position, and it is
desired to move it back to the datum position.  This can be done using
the following method:

\begin{minted}{python}
gd.reverseMotion(grid_state)
\end{minted}

Reversing the waveform requires that there is, firstly, a current
waveform configured, and, secondly, that this waveform remains valid.
After any normal, successful movement, a waveform remains valid for
repetition or reversal. After any collision, limit switch breach, or
abortion of a movement, the waveform becomes invalid. Generally
spoken, a waveform becomes invalid when either, the movement it
describes was interrupted and not completed, or when the step counters
become invalid (for example due to a collision).


\subsection{Repeating a movement}

Sometimes, a waveform for moving in a certain direction has been
configured, and this movement just needs to be repeated.  To do this,
use the following code:

\begin{minted}{python}
gd.repeatMotion(grid_state)
\end{minted}


\section{Recovery from collisions and limit breaches}
\label{sec:recovery}

Because the FPUs are controlled individually, and the CAN-level driver
has no concept of their geometric configuration, it cannot take care
to move them so that collisions are avoided.  Instead, the hardware
needs to handle collisions in a way that no damage results and
communicate such situations back to the higher layers of the software.

The following section explains how this is achieved.

\subsection{What happens when a collision or limit switch breach occurs}

Situations in which the movement of the hardware is obstructed
are either collisions between FPUs or limit switch breaches
of the alpha arm. Limit switch breaches are detected
when the alpha arm detects a transition where the
limit switch is on and goes off. All other cases are
considered collisions, which is detected by electrical
circuits which protect the beta arm. This includes
movements where the beta arm is moved out of its
allowed range.

When a collision or limit switch breach occurs, the FPU electronic
hardware stops any movement, and sends a message to the driver. On
receiving such a message, the driver registers the state of the FPU in
the internal \texttt{grid\_state} structure. This is reflected in the
status flags of the FPU, which were discussed above in section
\ref{sec:fpustate} on page \pageref{sec:fpustate}.  The \texttt{state}
field of the FPU is set to the value \texttt{OBSTACLE\_ERROR}, and the
flags \texttt{beta\_collision} and \texttt{at\_alpha\_limit} are set
accordingly.

If any movement operation or \texttt{findDatum()} command is going on,
this command returns with the updated state information on the
collision, and without further waiting for the movement to complete.
When the driver call returns to the Python wrapper, the error is
checked for and transformed into a Python exception.

It is also possible that a collision occurs while an FPU is \emph{not}
moving and the driver not executing a command.  In this case, the
internal \texttt{grid\_state} structure is changed, too. When the user
tries to launch a new command, the changed state is detected, and if
the command is not valid in collided state, the called method also
raises an exception.\footnote{Because movement methods return on the
  first collision, but in a multi-FPU grid additional collisions can
  occur after that, when controlling multiple FPUs it is a good idea
  to retrieve an updated grid state structure after handling any
  collision.}

The following recipe excludes the task of determining in which
direction FPUs should be moved after a collision.  In the general
case, this can be a complex question. The \texttt{grid\_state}
structure to provide a host of information for making this decision.

\subsection{Resolution of a beta arm collision}

In case of a collision of a beta arm, there are two
methods for resolving this. The following snippet
shows how to use them:

\begin{minted}[linenos]{python}
  from FpuGridDriver import REQD_CLOCKWISE, REQD_ANTI_CLOCKWISE
  fpu_id = 0
  gd.freeBetaCollision(fpu_id, REQD_CLOCKWISE, grid_state)
  # required for firmware version 1
  gd.pingFPUs(grid_state)
  gd.enableBetaCollisionProtection(grid_state)
\end{minted}

The effect of these methods is as follows:

In line 1, the symbols \texttt{REQD\_CLOCKWISE} and
\texttt{REQD\_ANTI\_CLOCKWISE} are imported.
Then, the parameter \texttt{fpu\_id} is set to the
numerical ID of the FPU. Then, calling the method
\texttt{freeBetaCollision()} moves the collided FPU
into the direction passed in the second parameter
- clockwise, or anti-clockwise. When using the protocol
version 1, it is then necessary to retrieve an update
for position and state of the FPU using the \texttt{pingFPUs()}
command.

This might need to be repeated a few times, and verified using visual
inspection, or camera pictures. When the collision is resolved, the
FPU can be switched to the normal state using the
\texttt{enableBetaCollisionProtection()} command.  After this, the FPU
can be moved normally. Because any previously configured waveforms
become invalid, a new movement needs to be configured using
\texttt{configMotion()} at this point.

\subsection{Resolution of an alpha limit switch breach}

For firmware and CAN protocol  version 2, limit switch
breaches can be handled in an analogous way as the
beta arm collisions. For CAN protocol version 1,
is option is not available. Instead, the following
steps are required:

\begin{minted}[linenos]{python}
  # Determine and write down the last movement of the FPU before
  # running into the limit switch. Let's assume
  # the alpha arm was moved with a positive sign.

  # reset the FPU to allow for movements
  gd.resetFPUs(grid_state)

  # move the FPU into the opposite direction
  waveform = gen_wf(-5, 0)
  gd.configMotion(waveform, grid_state, check_protection=False)
  gd.executeMotion(grid_state)
  # A second limit switch breach exception happens here!

  # reset the FPU again, to clear the second breach
  gd.resetFPUs(grid_state)
  gd.configMotion(waveform, grid_state, check_protection=False)
  gd.executeMotion(grid_state)

  gd.findDatum(grid_state)
  
\end{minted}
  

What these operations do is basically to clear the breach state, and
to manually move the FPU.  The flag \texttt{check\_protection=False}
in line 10 is required because normally, the software does not accept
a movement of an FPU which was not zeroed by a datum operation.

However, in this state, we \emph{must not} use a datum operation,
therefore it is necessary to override this software protection.

After this, the FPU can be moved, but because it crosses
the limit switch a second time, another limit switch
message is generated, and the FPU needs to be moved
manually for a second time.

\textbf{It is very important that, when resolving such a limit switch
  breach, the FPU is not moved into the same direction as it was
  moving before.  Otherwise, the FPU would run into the hard stop, and
  there is no hardware or software protection which could prevent
  damage in this case}.


\subsection{Limitations of CAN protocol version 1}

The following are the main limitations and functional
differences for protocol version 1:

\begin{itemize}
\item There is no automatic recovery method for alpha
  limit switch breaches. Limit switch breaches need
  to be recovered manually, and it must be carefully
  observed to move the FPU in the correct direction,
  on order to prevent the risk of damage. Espacially,
  a datum search must not be used after an alpha limit switch
  breach.

\item The current position and direction of movement of the FPUs is
  not tracked during ongoing movements, and the driver cannot record
  the last movement direction. This has the effect that the user needs
  to verify manually in which direction the FPU was moving before
  resolving a collision or limit switch breach.

\item In protocol version 2, transitions and allowed commands are
  checked much stricter than in protocol 1.  This results in a
  somewhat reduced flexibility, however the stricter checking also
  allows to make stronger assumptions about the current state, and to
  perform automatic recovery of multiple collided FPUs.
  
\item In some cases, protocol 1 is not able to
  detect lost messages, so that state information
  might be stale.\footnote{For the case of the \texttt{executeMotion()}
  command, a state update is performed automatically
  if no error has occurred.}

  
\item Uploading waveforms might take a longer time, especially if the
  waveform has a large number of steps.
  
\end{itemize}


\section{Reference}
\subsection{Commands}
\subsection{Errors}
\subsection{State Machine}


\end{document}
